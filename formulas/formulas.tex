\documentclass[12pt,a4paper]{extarticle}
\usepackage[utf8]{inputenc}
\usepackage[spanish]{babel}
\usepackage{amsmath}
\usepackage{amsfonts}
\usepackage{amssymb}
\usepackage[
  left= 1 cm,
  right= 1 cm,
  top= 1 cm,
  bottom= 2 cm
]{geometry}

\usepackage{graphicx}

% \title{Fórmulas de Física - Resumen Final}
% \author{Camila Nanini}


\begin{document}

% \maketitle

\section{Trigonometría y Vectores}
\includegraphics[width=1\textwidth]{FISICA2.png}
\begin{itemize}
    \item $180^\circ = \pi \text{ Rad}$ \textbf{Com Vector:}$A_x = A \cos \theta , A_y = A \sin \theta$,$|\vec{A}| = \sqrt{A_x^2 + A_y^2}$ , $\vec{A} \cdot \vec{B} = |\vec{A}||\vec{B}| \cos(\alpha)$
\end{itemize}

\section{Cinemática}

\begin{itemize}
    \item \textbf{Distancia entre dos puntos:} $d = \sqrt{(x_2 - x_1)^2 + (y_2 - y_1)^2}$
    \item \textbf{Velocidad Media:} $v_{med} = (x_f - x_i)/(t_f - t_i) $
    \item \textbf{Aceleración Promedio:} $a_{prom} = (v_{xf} - v_{xi})/(t_f - t_i)$
    \item \textbf{Mov. Rectilíneo Uniforme (MRU):}
    $a = 0 \implies v = \text{constante} \implies x(t) = x_0 + v \cdot t$
    \item \textbf{Mov. Rect. Uni. Variado (MRUV):}
    $a = \text{cte} \implies v(t) = v_0 + a \cdot t \implies x(t) = x_0 + v_0 t + \frac{1}{2} a t^2$
    \item \textbf{Movimiento Circular:}
    \begin{itemize}
        \item Período ($T$): Tiempo por vuelta. $T = (2\pi R) / v$
        \item Frecuencia ($f$): $f = 1/T$ (Hertz, Hz).
        \item Aceleración centrípeta: $a_c = v^2 / r = \omega^2 r$
        \item Vector de Posición: $\vec{r}(t) = R[\cos(\theta(t))\hat{i} + \operatorname{sen}(\theta(t))\hat{j}] \implies |\vec{r}(t)| = R$
        \item Velocidad Angular: $\omega(t) = d \theta(t)/{dt}$
        \item Vector Velocidad: $\vec{v}(t) = R\omega(t) [-\operatorname{sen}(\theta(t))\hat{i} + \cos(\theta(t))\hat{j}] \implies |\vec{v}(t)| = R|\omega(t)|$
        \item Aceleración Angular: $\gamma(t) = d\omega(t)/{dt}$
        \item Vector Aceleración: $\vec{a}(t) = -R\omega^2(t) [\cos(\theta(t))\hat{i} + \operatorname{sen}(\theta(t))\hat{j}] + R\gamma(t) [-\operatorname{sen}(\theta(t))\hat{i} + \cos(\theta(t))\hat{j}] = -R\omega^2(t) \hat{r}(t) + R\gamma(t) \hat{u}_v(t) = \vec{a}_n(t) + \vec{a}_t(t)$
    \end{itemize}
\end{itemize}

\section{Dinámica y Cantidad de Movimiento}

\begin{itemize}
    \item \textbf{Leyes de Newton:}
    $ \Sigma \vec{F} = 0 \iff \text{Equilibrio} (\vec{v} = \text{constante})
    , \Sigma \vec{F} = m \vec{a} , \vec{F}_{a,b} = -\vec{F}_{b,a}$
    \item \textbf{Fuerza y Momento Lineal:}
    $\vec{F} = \frac{d\vec{P}}{dt} \quad \text{donde} \quad \vec{P} = m\vec{v}$
    \item \textbf{Impulso Lineal:} $\vec{I} = \int_{t_1}^{t_2} \vec{F} dt = \Delta \vec{P}$
    \item \textbf{Rozamiento:}
    $F_{re} \le \mu_e N \quad \text{y} \quad F_{rd} = \mu_d N$
    \item \textbf{Ley de Hooke (Resortes):} $\vec{F} = -k \Delta x$
    \item Movimiento Oscilatorio Armónico (MOA):
    \begin{itemize}
        \item A = amplitud con respecto al punto de equilibrio (el punto que estaría en reposo)
        \item $x(t) = A \cos(\omega t + \varphi), v(t) = -A\omega \sin(\omega t + \varphi), a(t) = -A\omega^2 \cos(\omega t + \varphi)$
        \item $\omega = 2\pi / T = 2\pi f , T = \frac{1}{f}$,$v_{\max} = A\omega , a_{\max} = A\omega^2$
        \item MOA masa–resorte: $\omega = \sqrt{k/m} , T = 2\pi \sqrt{m/k}$
        \item MOA péndulo simple (ángulos pequeños) : $\omega = \sqrt{g/\ell} , T = 2\pi \sqrt{\ell/g}$
    \end{itemize}
\end{itemize}

\section{Trabajo y Energía}

\begin{itemize}
    \item \textbf{Trabajo ($W$):} $W = F \Delta r \cos(\theta)$ (fuerza cte), $W = \int \vec F \cdot d\vec r$ (fuerza no cte)
    \item \textbf{Energía Cinética ($K$):} $K = (m v^2) / 2$
    \item \textbf{Teorema Trabajo-Energía Cinética:} $W_{tot} = \Delta K = K_f - K_i$
    \item \textbf{Fuerzas Conservativas:} $W_c = -\Delta U = U_i - U_f$
    \begin{itemize}
        \item Energía Potencial Gravitatoria: $U_g = m \cdot g \cdot h$
        \item Energía Potencial Elástica: $U_e = (k \Delta x^2) / 2$
    \end{itemize}
    \item \textbf{E. Mecánica:} $E = K + U$. Solo conservativas: $\Delta E = 0$ y con no conservativas: $\Delta E = W_{NC}$
\end{itemize}

\section{Momento Lineal, Impulso y Choques}

\begin{itemize}
    \item \textbf{Teorema Impulso-Cantidad de Movimiento:} $$\Delta \vec{P} = \int_{t_i}^{t_f} \vec{F} dt \equiv \vec{I}$$.
    \item Sistema aislado : se conserva P. El mismo antes y despues del choque.
    
    \item \textbf{Choque Inelástico (Plástico):}
    \begin{itemize}
        \item Dsp del choque las partes quedan unidas en un solo cuerpo y $K_f < K_i$ (Ej:plastilina vs caja)
        \item \textbf{Energía inicial del choque plástico:} $$K_i = \frac{1}{2} m_1 v_{1i}^2 + \frac{1}{2} m_2 v_{2i}^2$$.
        \item \textbf{Coeficiente de Restitución ($e$):} $e = \frac{v_{2f} - v_{1f}}{v_{1i} - v_{2i}}$, en general ($e < 1$) pero si es totalmente Inelástico ($e = 0$)
        \item \textbf{Velocidad final masa 1:} 
        $$ v_{1f} = \frac{m_1 v_{1i} + m_2 v_{2i} - m_2 e (v_{1,i} - v_{2i})}{m_1 + m_2}$$
        \item \textbf{Velocidad final masa 2:}
        $$ v_{2f} = \frac{m_1 v_{1i} + m_2 v_{2i} + m_1 e (v_{1i} - v_{2i})}{m_1 + m_2}$$
    \end{itemize}

    \item \textbf{Choque Elástico ($e = 1$):}
    \begin{itemize}
        \item Si la fuerza entre las partes del sistema son conservativas entonces se conserva la E. mecanica(Ej:bolas de billar) 
        \item \textbf{Energía Cinética Inicial ($K_i$):} $$K_{i}=\frac{1}{2}m_{1}v_{1i}^{2}+\frac{1}{2}m_{2}v_{2i}^{2}$$.
        \item \textbf{Energía Cinética Final ($K_f$):} $$K_{f}=\frac{1}{2}m_{1}v_{1f}^{2}+\frac{1}{2}m_{2}v_{2f}^{2}$$.
        \item \textbf{Relación de Velocidades:} $v_{1i} - v_{2i} = -(v_{1f} - v_{2f})$.
        \item \textbf{Velocidad final masa 1 ($v_{1f}$):} 
        $$v_{1f} = \frac{(m_{1}-m_{2})v_{1i} + 2m_{2}v_{2i}}{m_{1} + m_2}$$
        \item \textbf{Velocidad final masa 2 ($v_{2f}$):} 
        $$v_{2f} = \frac{2m_{1}v_{1i} + (m_{2}-m_{1})v_{2i}}{m_{1} + m_2}$$ 
    \end{itemize}
\end{itemize}

\section{Electrostática}

\begin{itemize}
    \item \textbf{Ley de Coulomb:}
    \[
    \vec{F}_{12} = k_e \frac{q_1 q_2}{r^2}\,\hat{r}_{12}
    \]
    (fuerza eléctrica ejercida por \(q_1\) sobre \(q_2\), donde \(\hat{r}_{12}\) es un vector unitario dirigido desde \(q_1\) hacia \(q_2\)).
    \begin{itemize}
        \item \(k_e = \dfrac{1}{4\pi\varepsilon_0} \approx 8{,}99 \times 10^9 \, \text{N}\cdot\text{m}^2/\text{C}^2\) y \(\varepsilon_0 \approx 8{,}85 \times 10^{-12} \, \text{C}^2/(\text{N}\cdot\text{m}^2)\)
    \end{itemize}

    \item \textbf{Principio de superposición de la fuerza electrostática:}
    \[
    \vec{F} = \sum_{i=1}^{n} \vec{F}_i,
    \qquad
    \vec{F}_i = k_e\, \frac{q\, q_i}{r_i^{2}}\, \hat{r}_i
    \]

    \item \textbf{Campo eléctrico de una carga puntual:}
    \[
    \vec{E} = k_e \frac{q}{r^2}\,\hat{r}
    \]
    (campo eléc. gen. x una carga \(q\), donde \(\hat{r}\) apunta desde la carga a el punto donde se evalúa el campo).

    \item \textbf{Relación entre fuerza y campo eléctrico:}
    $\vec{F} = q\,\vec{E}$


    \item \textbf{Distribución continua de carga:}
    \[
    \vec{E} = k_e \int \frac{dq}{r^2}\,\hat{r}
    \]
    \begin{itemize}
        \item \(dq\): elemento infinitesimal de carga.
        \item \(r\): distancia desde \(dq\) al punto donde se calcula \(\vec{E}\).
        \item \(\hat{r}\): vector unitario que apunta desde \(dq\) hacia el punto de observación.
        \item \(k_e = \dfrac{1}{4\pi\varepsilon_0}\).
        \item Antes de integrar, \(dq\) debe expresarse en función de la densidad de carga:
        Lineal: \(dq = \lambda\, dl\) , Superficial: \(dq = \sigma\, dA\) , Volumétrica: \(dq = \rho\, dV\)
   
    \end{itemize}

    \item \textbf{Ley de Gauss:}
    \[
    \phi_E = \oint E \cos \phi \cdot dA = \oint \vec{E} \cdot d\vec{A} = \frac{Q_{\text{enc}}}{\varepsilon_0}
    \]
    \begin{itemize}
        \item \(Q_{\text{enc}}\): carga total encerrada por la superficie gaussiana.
        \item \(\varepsilon_0\): permitividad del vacío.
        \item \(\phi_E\): flujo eléctrico a través de la superficie gaussiana.
    \end{itemize}
    \begin{itemize}

    \item \textbf{Caso: Esfera maciza uniformemente cargada (aislante)}
    \begin{itemize}
        \item Interior (\(r<R\)):
        \[ Q_{\text{enc}} = Q\frac{r^3}{R^3}
        , E(r)=\frac{1}{4\pi\varepsilon_0}\frac{Q}{R^3}r\]

        \item Exterior (\(r\ge R\)):
        \[E(r)=\frac{1}{4\pi\varepsilon_0}\frac{Q}{r^2}\]
    \end{itemize}

    \item \textbf{Caso: Esfera conductora cargada}

    \begin{itemize}
        \item Interior (\(r<R\)):
        \[ Q_{\text{enc}}=0 \quad\Rightarrow\quad E=0\]

        \item Exterior (\(r\ge R\)):
        \[ E(4\pi r^2)=\frac{Q}{\varepsilon_0} \quad\Rightarrow\quad
        E(r)=\frac{1}{4\pi\varepsilon_0}\frac{Q}{r^2}
        \]
    \end{itemize}

    \item \textbf{Caso: Carga lineal infinita}
    \[ Q_{\text{enc}}=\lambda L \]
    \[ \quad E(r)=\frac{\lambda}{2\pi\varepsilon_0 r} \]

    \item \textbf{Caso: Lámina infinita cargada}
    \[ Q_{\text{enc}}=\sigma A \]
    \[ \quad E=\frac{\sigma}{2\varepsilon_0} \]
\end{itemize}

\item \textbf{Carga puntual que se mueve en un campo eléctrico uniforme:}
\[ W_{A\to B} = - \Delta U = q\int_A^B \vec E \cdot d\vec r = q_0Ed\]
es el caso de una carga puntual en el campo la energía potencial U = $q_0Ey$ donde y es la distancia vertical.Es el trabajo realizado por el campo eléctrico cuando una carga se mueve de A a B

\item \textbf{Energía potencial eléctrica de dos cargas puntuales:}
\[ U = k_e \frac{q q_0}{r} \]
o sea la carga de prueba q0 se desplaza en el campo generado por q.

\item \textbf{Energía potencial eléctrica de un sistema de dos o más cargas puntuales:}
\[U = k_e q_0 \sum_{i=1}^{n} \frac{q_i }{r_i}\]

\item \textbf{Potencial electrico:} El potencial eléctrico indica cuánta energía por unidad de carga tendría una carga de prueba si estuviera en ese punto.
\[V = \frac{U}{q_0} \rightarrow U = q_0 V\]

\item \textbf{Potencial debido a una carga puntual:} $V = k_e \frac{q}{r}$

\end{itemize}

\section{Capacitancia y Corriente Eléctrica}
\begin{itemize}
    \item \textbf{Capacitancia ($C$):} $C = Q / \Delta V$
    \item \textbf{Capacitancia de un capacitor de placas paralelas:} $C = \epsilon_0 A / d$
    \item \textbf{Capacitancia de un capacitor lleno de un material con constante dieléctrica $\kappa$:} 
    $C = Q_0 / \Delta V = \kappa Q_0 / \Delta V_0 $
    \item \textbf{Energía almacenada en un capacitor:} $U = (C V^2) / 2$
    \item \textbf{Asociación de capacitores:}
    \begin{itemize}
        \item Serie: $\frac{1}{C_{eq}} = \frac{1}{C_1} + \frac{1}{C_2} + \dots$ las diferencias de potencial se suman.
        \item Paralelo: $C_{eq} = C_1 + C_2 + \dots$ los capacitores tienen el mismo potencial V.
    \end{itemize}
    \item \textbf{Resistencia ($R$):} $R = \Delta V / I$ , Ley de Ohm: $\Delta V = IR$, resistencia eléctrica de conductor homo $R = \rho L/A$
    ,\(L\) es la longitud del conductor y \(A\) es el área de la sección transversal
    \item \textbf{Asociación de resistencias:}
    \begin{itemize}
        \item Serie: $R_{eq} = R_1 + R_2 + \dots$ la diferencia de potencial se suma.
        \item Paralelo: $\frac{1}{R_{eq}} = \frac{1}{R_1} + \frac{1}{R_2} + \dots$ la diferencia de potencial es la misma en cada resistencia.
    \end{itemize}
    \item La potencia disipada en un resistor: $P = I^2 R = \Delta V^2 / R$
    \item \textbf{Leyes de Kirchhoff:}
    \begin{itemize}
        \item Ley de nudos:En cualquier nodo $\sum I_{in} = \sum I_{out}$
        \item Ley de mallas: En cualquier ciclo $\sum \Delta V = 0$ 
    \end{itemize}
    \item \textbf{Carga y corriente en función del tiempo} si un capacitor $C$: \\
    Se carga con una fem $\varepsilon$ a través de una resistencia $R$:
        \[ Q(t) = C\varepsilon(1 - e^{-t/\tau}) \quad \quad I(t) = \frac{\varepsilon}{R} e^{-t/\tau} \]

    Se descarga a través de una resistencia $R$:
        \[ Q(t) = Q_0 e^{-t/\tau} \quad \quad I(t) = -\frac{Q_0}{\tau} e^{-t/\tau} \]

    Con $\tau = RC$ (luego de este tiempo el capacitor está cargado en un 63,2 de la capacitancia)
\end{itemize}

\section{Magnetismo}
\begin{itemize}
    \item Fuerza sobre una carga \(q\) con velocidad \(\vec{v}\) en un campo magnético \(\vec{B}\) es: $\vec{F} = q\,\vec{v} \times \vec{B}$
    \item Theta es el ángulo entre v y B. Entonces F = 0 cuando v es paralela/antiparalela a B (0 o 180°) y es máxima cuando v es perpendicular a B (90°). $F = |q| v B \sen \theta$
    \item Si la partícula se mueve en una región donde hay campos eléctrico y magnético: $\vec{F} = q\left(\vec{E} + \vec{v} \times \vec{B}\right)$
    \item Un conductor rectilíneo de long l, por el cual circula una corriente \(I\), en un campo mag \(\vec{B}\), experimenta la siguente $\vec{F}_B$:$\vec{F}_B = I\,\vec{l} \times \vec{B}$
    La magnitud de la fuerza es \(F_B = I\,l\,B\,\sin\phi\), con \(\phi\) es el ángulo entre el conductor y el campo magnético y \(\vec{l}\) es un vector de modulo = l y dirección de la corriente.
    \item (\textit{Regla de la mano derecha}): pulgar apunta en dirección de la velocidad de la carga + o de I; los dedos extendidos en dirección del campo magnético y la palma señala la dirección de la fuerza magn.
    \item Flujo mag. a través de una sup.: $\Phi_B = \int B_\perp\, dA = \int B \cos\phi\, dA = \int \vec{B}\cdot d\vec{A}$. Si la sup. es cerrada = 0. 
    \item Ac. centrípeta: \(a_c = \dfrac{v^2}{R}\) y la única fuerza que actúa es la magnética. Por Newton: $|\vec{F}| = |q|\,vB = m\,\frac{v^2}{R}$
    \item Radio de una órbita circular en un campo magnético : $R = (m v) / (|q| B)$. 
    \item Rapidez angular: $\omega = v / R$ 
    \item Frecuencia del ciclotrón es: $f = \omega/ (2\pi)$
    \item La magnitud del par de torsión que actúa sobre una espira de corriente en un campo magnético es: $\tau = I B A \sin\phi$
    y en forma vectorial: $\vec{\tau} = \vec{\mu} \times \vec{B}$
    \item Momento magnético de la espira: $\vec{\mu} = I\,\vec{A}$ . Con N espiras de la misma área $\vec{\mu}_{bobina} = N I\,\vec{A}$
    \item La E pot. de un dipolo magnético en un campo magnético está dada por: $U = -\vec{\mu}\cdot\vec{B} = -\mu B\cos\phi$
    \item (\textit{Ley de Biot--Savart}):
    \[ d\vec{B} = \frac{\mu_0}{4\pi}\,\frac{I\,d\vec{l} \times \hat{r}}{r^2}\]
    con \(\hat{r}\) vector unitario que apunta desde el elemento de corriente hasta el punto de observación y \(r\) es la distancia entre ellos y la permeabilidad del espacio libre es $\mu_0 = 4\pi \times 10^{-7}\ \text{T·m/A}$
    \item Una carga \(q\) con velocidad \(\vec{v}\) genera un campo magnético en un punto situado a una distancia \(r\):
    \[
    \vec{B} = \frac{\mu_0}{4\pi}\,\frac{q\,\vec{v} \times \hat{r}}{r^2}
    \]
    donde \(\hat{r}\) es el vector unitario que apunta desde la carga hacia el punto de observación.
    \item Campo mag de un alambre: $(\mu_0 I) / (2\pi r )$ (con r la distancia entre el alambre y el pto que considero)
    \item $F_B$ entre dos alambres paralelos (\(a\) es la distancia entre ellos y l long de los alambres):
    $$\frac{F_B}{\ell} = \frac{\mu_0 I_1 I_2}{2\pi a}$$ 
    \item (\textit{Ley de Ampère}) La integral de línea del campo magnético \(\vec{B}\) a lo largo de cualquier trayectoria cerrada es igual a \(\mu_0\) por la corriente total encerrada: $\oint \vec{B}\cdot d\vec{s} = \mu_0 I$
    \item Campo magnético de un toroide: $$B = \frac{\mu_0 N I}{2\pi r}$$ y un solenoide : $B = \mu_0 n I$ donde \(n = \dfrac{N}{\ell}\) es el número de vueltas por unidad de longitud.
    \item (\textit{Ley de inducción de Faraday}) $\mathcal{E} = - (d\Phi_B) / dt$ 
    \item $\Phi_B = \int_S \vec B(\vec r,t)\cdot d\vec A$ con (S): una superficie cualquiera, ($d\vec A$): vector normal a la superficie, de módulo (dA) ($\vec B(\vec r,t)$) campo magnético, que depende del tiempo y posicion
    \item (\textit{FEM de mov.}) Cuando un conductor de longitud \(l\) se mueve con velocidad \(\vec{v}\) en un campo magnético \(\vec{B}\), la fuerza electromotriz inducida es:
    $\mathcal{E} = B\,l\,v$ (caso de mov. perpendicular al campo magnético).
    \item La corriente inducida en un circuito de resistencia \(R\): $I = \mathcal{E}/R$
    \item Potencia disipada por una resistencia \(R\) debido a la corriente inducida:
    $P = I^2 R = \mathcal{E}^2/R$
\end{itemize}

\section{Termodinámica}
\begin{itemize}
    \item Dos objetos están en \textbf{contacto térmico} si pueden intercambiar energía
    entre ellos. Y están en \textbf{equilibrio térmico} si están en contacto térmico y no hay
    intercambio neto de energía entre ellos.
    \item \textbf{Celsius a Fahrenheit:} $T(^\circ\text{F}) = \frac{9}{5} T(^\circ\text{C}) + 32$
    \item \textbf{Celsius a Kelvin:} $T(\text{K}) = T(^\circ\text{C}) + 273.15$
    \item \textbf{Expansión Lineal:} 
    $ \Delta L = \alpha L_0 \Delta T \quad \text{o bien} \quad L - L_0 = \alpha L_0(T - T_0) $
    . Con $L$ longitud final, $T$ temperatura final y $\alpha$ es el coeficiente de expansión lineal en unidades $(^\circ\text{C})^{-1}$.
    \item \textbf{Expansión de Área:}  $ A = A_0(1 + 2\alpha \Delta T) \quad \text{o bien} \quad A = A_0 + 2\alpha A_0 \Delta T $
    \item \textbf{Cambio de Área:} $ \Delta A = A - A_0 = \gamma A_0 \Delta T $
    . Donde $\gamma = 2\alpha$ es el coeficiente de expansión de área.
    \item \textbf{Expansión Volumétrica:} $ \Delta V = \beta V_0 \Delta T $. Con $\beta$ coeficiente de expansión volumétrica y $\beta = 3\alpha$.
    \item \textbf{Calor requerido:} $ Q = mc \Delta T $. Donde $\Delta T = T_2 - T_1$ es el cambio de temperatura y $c$ es el calor específico.
    $ C = mc $ Representa la cantidad de calor necesaria para elevar un grado la temperatura de toda la masa del objeto. 
    A veces en vez de la masa se usa la cantidad de sustancia ($n$) y la masa de sus moles ($M$) en ese caso $ m = nM $
    \item \textbf{Calor de cambio de fase:} 
    $ Q = \pm mL $. Se usa (+) si el material se vaporiza (entra calor) y (-) si se congela (sale calor).
    \textit{Nota: Durante el cambio de fase no hay variación de temperatura.}
    \begin{itemize}
        \item \textbf{Constantes para el agua:}
        \item Fusión: $L_f = 3.34 \times 10^5 \text{ J/kg} = 79.6 \text{ cal/g} = 333 \text{ kJ/kg}$
        \item Vaporización: $L_v = 2.256 \times 10^6 \text{ J/kg} = 539 \text{ cal/g} = 2256 \text{ kJ/kg}$
    \end{itemize}
    \item \textbf{Ecuación de los Gases Ideales:} $PV = nRT $. Donde:
    $P$ es la presión = Modulo de Fuerza/Área. $V$ es el volumen. $n$ es el número de moles.
    $R$ es la constante universal de los gases ($8,314 J/\text{mol}\cdot\text{K}$). $T$ es la temperatura absoluta en Kelvin ($K$).
    \item \textbf{Ley de Boyle:} A temperatura constante, la presión de un gas es inversamente proporcional a su volumen. O sea: $P = \frac{1}{V}$.
    \item \textbf{Ley de Charles:} A presión constante, el volumen de un gas es directamente proporcional a su temperatura absoluta. Matemáticamente: $\dfrac{V}{T} = \text{cte}$.
    \item Masa constante de gas ideal, el producto $nR$ es constante y $\dfrac{pV}{T}$ tmb. Si los subíndices $1$ y $2$ son dos estados distintos entonces:
    $$\frac{p_1 V_1}{T_1} = \frac{p_2 V_2}{T_2} = \text{cte}$$
    \item Procesos: isotérmicos = T ctate , isobéricos = P ctate , isocóricos = V ctate, adiabáticos = aislados
    \item \textbf{Número de Avogadro ($N_A$):} Cantidad de moléculas en un mol: $N_A = 6.022 \times 10^{23} \, \text{mol}^{-1}$
    \item \textbf{Masa Molar ($M$):} Es la masa de un mol de sustancia, equivalente a la masa de una molécula ($m$) por el número de Avogadro: $M = N_A m $
    \item Convenciones: Calor positivo entra al sistema y negativo si sale. Trabajo positivo cuando el sistema realiza trabajo sobre el entorno y negativo cuando el entorno realiza trabajo sobre el sistema.
    \item En un cambio finito de volumen desde $V_1$ hasta $V_2$, el trabajo realizado está dado por
    $$W = \int_{V_1}^{V_2} p\, dV$$
    \item \textbf{En un proceso isobárico:} El trabajo realizado por el gas se calcula como 
    $$W = \int_{V_1}^{V_2} p\, dV = p\,(V_2 - V_1)$$
    \item \textbf{Primera Ley de la Termodinámica:}
    $ \Delta U = Q - W $
    \item Trabajo isotérmico: $W = nRT \ln \left( \frac{V_f}{V_i} \right)$
    
\end{itemize}

\section{Unidades}
Velocidad: \(\mathrm{m/s}\)
     Aceleración: \(\mathrm{m/s^2}\)
     Newton: \(\mathrm{N} = 1\,\mathrm{kg}\cdot\mathrm{m/s^2}\)
     Joule: \(\mathrm{J} = 1\,\mathrm{N}\cdot\mathrm{m}\)
     Wats:\(\mathrm{W} = 1\,\mathrm{J/s}\)
     Volt: \(1\,\mathrm{V} = 1\,\mathrm{J/C}\)
     Faradio: \(1\,\mathrm{F} = 1\,\mathrm{C^2/(N\,m)}\)
     Ampere: \(1\,\mathrm{A} = 1\,\mathrm{C/s}\)
     Ohm: \(1\,\Omega = 1\,\mathrm{V/A}\)
     Tesla: 
\[
1\,\mathrm{T} = \frac{\mathrm{V\,s}}{\mathrm{m^2}} 
= \frac{\mathrm{J}}{\mathrm{A\,m^2}} 
= \frac{\mathrm{N\,s}}{\mathrm{C\,m}} 
= \frac{\mathrm{kg}}{\mathrm{C\,s}} 
= \frac{\mathrm{N}}{\mathrm{A\,m}}
\]



\end{document}